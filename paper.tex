\documentclass[sigconf]{acmart}
\settopmatter{printacmref=false} % Removes citation information below abstract
\renewcommand\footnotetextcopyrightpermission[1]{} % removes footnote with conference information in first column

\usepackage[UTF8]{ctex}
\usepackage{booktabs} % For formal tables
\usepackage{url}
\usepackage{amsmath}
\usepackage{amssymb}
\usepackage{dirtytalk}
\usepackage{graphicx}
\usepackage{hyperref}
\usepackage{wrapfig}
\usepackage{flushend} % 参考文献充实堆积
\usepackage{listings}
\usepackage{color}

\definecolor{dkgreen}{rgb}{0,0.6,0}
\definecolor{gray}{rgb}{0.5,0.5,0.5}
\definecolor{mauve}{rgb}{0.58,0,0.82}

\renewcommand{\vec}[1]{\mathbf{#1}}

\lstset{frame=tb,
	language=Python,
	aboveskip=3mm,
	belowskip=3mm,
	showstringspaces=false,
	columns=flexible,
	basicstyle={\small\ttfamily},
	numbers=none,
	numberstyle=\tiny\color{gray},
	keywordstyle=\color{blue},
	commentstyle=\color{dkgreen},
	stringstyle=\color{mauve},
	breaklines=true,
	breakatwhitespace=true,
	tabsize=3
}

\setcopyright{none}

\acmConference[]{}{}{}
\acmYear{2017}
\copyrightyear{2017}

\begin{document}
\title{Chaucer's Art in The Wife of Bath’s Tale}

\author{林学渊16343049}
\affiliation{
	\department{荔园11号418}
	\institution{School of Mathematics (Zhuhai), Sun Yat-sen University}
}

\begin{abstract}
In “The Wife of Bath’s Tale” written by Geoffrey Chaucer, “Tale of Florent” in “Confessio Amantis” written by Chaucer's contemporary, friend and poet, John Gower(1330-1408) and “The Wedding of Sir Gawain and the Dame Ragnell” believed to written by Sir Thomas Mallory, we see the similar tales that hold the same message. While the tales hold the same message their contents are very different. To discover why these changes were made by Chaucer we observe the works in depth. Also, we look at the origins of these stories, and how they inspired the changes. In the end, we draw a conclusion that the reason are literary characters, narratives, and what women desire.
\end{abstract}

\keywords{Narratives; Tales; Literary characters; Marriage; Weddings; Women Rights}

\maketitle

%\noindent {\bf Contents}
%
%\smallskip
%\S\ref{Introduction}. Introduction
%
%\S\ref{tale}. The Wife of Bath's Tale
%
%\S\ref{prototype}. The Tale's Prototype
%
%\S\ref{Creative changes}. Creative changes
%
%\S\ref{Analysis}. Analysis
%
%\S\ref{Conclusion}. Conclusion
%\medskip

\section{Introduction}
\label{Introduction}

Lively, absorbing, often outrangeously funny, Chaucer's “The Canterbury Tales”\cite{chaucer2008canterbury} is a work of genius, an undisputed classic that has held a special appeal for each generation of readers. The “Tales” gathers twenty-nine of literature's most enduring (and endearing) characters in a vivid group portrait that captures the full spectrum of medieval society, from the exalted Knight to the humble Plowman.

THE WIFE OF BATH is perhaps the most memorable of the pilgrims on the road to Canterbury. She is memorable in part simply because we know about any other pilgrim. In addition to the thirty-two-line description of her in the General Prologue to “The Canterbury Tales” (see lines 447-78), we have the amazing 856-line prologue to her tale in which she gives a confessional account of her life, her philosophy of love and marriage, and her various husbands. But Alisoun of Bath is memorablemostly just because of who she is: a wonderfully varied, funny,  garrulous, abrasive, pushy, honest, dishonest, hateful, lovable, and contradictory woman who knows how to yiled, flowerlike, to get what she wants from men, but also how to stand up to them, bulldozer-like, when she needs to.

"The Wife of Bath's Tale" is  all about "what thing it is that women most desire". Intriguingly, Chaucer made a very clever inheritance and extension of the folk literature and literati texts. The story is extremely interesting. The essence of relationship between men and women is deeply involved. Inside, the "sovereignty" and "gentillesse" have always been the core value of Western culture.\cite{owst1933literature}

\section{The Wife of Bath's Tale}
\label{tale}
The story took place in King Arthur's time, which is about an Arthur's knight raped a female citizens on the way hunting falcon. The woman sued to the court. According to the law, the knight should have executed, but Queen pleaded with Arthur to thank and allowed her to dispose the knight. The Queen proposed that she could save the knight from death, providing that he had to find the answer within one year and one day, the usual time frame in medieval times, about what thing it was that woman most desire? The knight asked around, but the answer varied. Some people said that women loved money, some said that women loved fame, some said that women loved to be happy, some said beautiful clothes, some  said the desire on the bed, some people said women wanted to most oftenly change herselves from widows to remarry, some people said that women loved men flattering. Some people said that women loved freedom, doing what they wanted to do, did not like to listen to men or to be honest, because though a woman even had the heart of evil, she still wanted to be said to be a pure wisdom. Others said that a woman loved man felt she was loyal and trustworthy, though she can not even keep secrets at all ...\footnote{"Some saiden wommen loven best richesse; Some saide honour, some saide jolinesse; Some riche array, some saiden lust abedde, And ofte time to be widwe and wedde. Some saide that oure hearte is most esed; And some sayen that we loven best, For to be free, and do right as us lest; And some sayn that greet delit han we, For us to be holden stable and eek secree" (Norton, 137, lines 931-935, 941-942, 951-952.).}

Deadline approaching, the knight was afraid to do nothing but return. When passing by the woods, he saw there were twenty four fairies dancing in the forest. He hurried forward, but found that there was only an old woman sitting alone. The old woman taught him the answer, but required him to fulfill one of her demands. The two then went back to the court. That day, Queen and ladies all presented. The knight gave the public the answer that the woman most wanted was sovereignty over men and domination over men. "Married women, unmarried women and widows" all agreed with this answer. Knight got his life, but immediately the old woman approached, asked Knight to meet her only request, married her. She wanted him not only to marry her, but also to love her. Knight had to obey in order to keep his promise. On the marriage bed, the groom was tossing and sighing. Bridal laughed and asked, Arthur court knight in wedding night was so hard to please? Where did I make a mistake? I could change it. Groom answered, "change? Can not change forever! Why are you so old and ugly?" The bride said, " If it is only for this, I can change, but what is your noble, what is poverty, what is the old? Noble and background, family property has nothing to do. Noble is not from inheritance, but only from God. Noble is all about the noble man. Why poor is wrong? What Jesus Christ chose was a poor, innocent life. Really poor is unlucky things, nothing. Heart is magnanimous. How about old? Do gentlemen should not respect the elderly? After this instruction, the woman again, "I know your mind. I will satisfy your secular desire. But you have to choose. You choose my old ugly but loyal, or choose me beauty but freedom? I will not betray you if I were an old lady, but if I were beautiful, I may." Knight thought, but he could not decide, said, "my master, my love, my dear wife, I give you the choice, how do you choose my choice, your choice is my choice." the old woman said, "You make me the Lord? Then kiss me! I need both, and I will be your beautiful and loyal wife forever. Set off the curtain to see me!" Knight turned around but found a most beautiful and cutest woman. He hugged her ecstatically, kissed her thousands of times ...\cite{aguirre1993riddle}


\section{The Tale's Prototype}
\label{prototype}
Most of medieval literature has prototype and "The Wife of Bath's Tale" is no exception. Actually, Most of the stories in “The Canterbury Tales” have their own origins, and “The Knight's Tale”, which has no origin, is not finished. Therefore, the so-called innovation, is the transformation, not the original, but the result is far better than the other versions of the story.

Although the prototype of "The Wife of Bath's Tale" is lost, several variations of this prototype can still test the original appearance. Several major factors in the story are quest, sovereignty, choice, enchantment and metamorphosis. In addition to the long age's, other countries', considering that if saying too much involved, it is inevitable to fetch, I cite only two examples in English literature. They are the closest in time, place, and content to "The Wife of Bath's Tale". They are Tale of Florent in "Confessio Amantis"\cite{gower1980confessio} written by Chaucer's contemporary, friend and poet, John Gower(1330-1408)\cite{fisher1964john} and "The Wedding of Sir Gawain and the Dame Ragnell"\cite{wedding1924weddynge} believed to written by Sir Thomas Mallory.

\subsection{Tale of Florent in Confessio Amantis}
Once Gower's Florent mistakenly killed a castle heir, whose grandmother asked him to correctly answer the question, "What Women Most Want", to save him from death. Florent found hard but found nothing, and finally met an elderly woman under the tree in the forest. Woman said she could save his life, but he should marry her. The answer was that a woman most wanted to control a man. The old grandmother was furious when she got the answer, saying that he must have been told by some, who were cursed to be burned to death.

On the wedding night, Floren laid in the bed and sighed. Suddenly he saw a bright light inside the room, surprised to see that he was lying next to an 18-year-old beauty. Beauty asked him to choose either she became young and beautiful in the daytime but old and ugly at night or became old and ugly during the day but young and beautiful at night? Florens could not decide, saying he would like to hear her choice. The woman thanked him for giving her the right and said, her stepmother made her like this and her curse could be broken only with the love and respect of the best knight. Now the facts proved that Florence was such a person taht she will always be young and beautiful.

\subsection{The Wedding of Sir Gawain and the Dame Ragnell}
“The Wedding of Sir Gawain and the Dame Ragnell” is similar in addition to some details. It tells a story about King Arthur was subject to a warriors (bold baron) when hunting and must answer the questions correctly, "What thing it is that women most desire", in order to regain freedom. An old ugly woman was willing to provide answers, but on condition that Gavin must marry her. Gavin, known for his loyalty, agreed to dedicate himself. When the time was up, Arthur said the correct answer, that is, the women most wanted to control men. Warrior was furious, saying that it must be his sister who leaked the information. he cursed his sister burned to death.

After the wedding, the old ugly woman became beautiful. But Gavin was asked to choose either she became young and beautiful in the daytime but old and ugly at night or became old and ugly during the day but young and beautiful at night? It is difficult for Gavin to make a decision, so that he was willing to listen to his wife choice. His wife said, “now that you let me choose, I choose to be beautiful and young day and night, forever. Beside, she said the bad stepmother made her such an apperance, turning her into such a ugly woman and could recover only with the love and respect of the first knight in England. Now not only had she change back to her original form, Gavin's behavior had also free his brother.
\section{Creative changes}
\label{Creative changes}
Compared with Tale of Florent and The Wedding of Sir Gawain and the Dame Ragnell, The Wife of Bath's Tale was made different in some details.

First, create a causality between the QUESTION and the FAULT the knight made, where there was no relationship in its original story at all. The knight was committed the crime of rape (men's greatest violation of women) instead of killing, leading to the unique punishment. The main thrust of this punishment is not killing, but to make the knight understand women. The judge is the queen, the leader of women in a country. He who gave the queen the penalty was the KING - suggesting that men should give women sovereignty and then men and women can enjoy harmony and contentment.

Second, cut off the story stubs and break the link between the question person and problem-solving people.

Third, remove the reason that because of the evil stepmother, beautiful woman was made an old woman. Therefore, the beauty could change herself without force, thus getting more independence and mobility. Besides, Chaucer also removes stepmother and brother and other more branches.

Forth, add the elderly woman's "noble" preaching in order to assist the knight's transformation. This is also Chaucer's unique addition.

So far the story of the moral significance shows up, and the knight completes the change from beast to human nature, form never wanting to marry the woman to take the initiative to call her "my master, my love, my kiss wife" and willing to give her the right to choose. Each part of this story is logical and with great accuracy, and fluency.

Another rare thing is that not only is this story the general rigor, but details are more subtle. It is worthy of a poet. Beginning with the ancient days of King Arthur, "This land was all filled with fairies. / The elf queen with her jolly company / danced often in many a green meadow". Then the knight saw twenty four fairies dancing gracefully in the forest. He then hurriedly approached but he only saw an elderly woman sitting under the tree in the forest, which is to respond to the beginning of this beautiful shy, while suggesting that the origin of an extraordinary woman. The fairy and the old woman caused drama contrast. In the end, the elder woman became back to beauty, making people guessing that she was the lively and happy fairy queen, and the total story was all about love game she directed. All such details are in few words. But it is so smooth and fluency.

\section{Analysis}
\label{Analysis}

Why do Chaucer remove so much from original story and make such change and creation?

It is because the narrator of the story is the wife of bath, and Chaucer wants her to give a story that "fits" her interesting identity.

Is the story fit? It really fits! However, it is not "submissive" fit, but "rebellious" fit. Because the Bath woman herself seemed rough, not matching romantic legend. She was a professional "woman", the most gluttonous woman, numerous lovers, married five times, still yearning sixth and more. From the age of twelve (the legal marriage age of medieval churches) , old man and young wife -- the most common marital relationship in the Middle Ages - to old lady and young husband. she was 40 years old and her husband was twenty and finally reached her most satisfying sexual relationship and power relations. Her vitality was extremely strong, demanding. Every husband died before she died. Some even suspected she had killed her fourth husband. The fifth twentieth was also dead, allowing her to aspire to the sixth. She was a "weaver" and a good textile. Her skill was no less than the most advanced Flanders weaver. She had room and money, and she was an experienced traveler who had been to Jerusalem three times, and also to Rome, Italy, Bologna, Spain, Galicia and Cologne, Germany. The purpose of these trips was possibly for fun while on the pilgrimage. She had a scarlet coat and stockings, which were symbols of sensuality and sin in the Bible. These have created her a most extreme feminist.

But do not consider Baths women has given up on patriarchy.\cite{steinberg1964wife}At the end of the story, just as the knight and the beautiful woman "lived to the end of their lives in perfect joy", the woman immediately returned to her noisy, noisy routine, "Jesus Christ send us /  husband who are meek, young, and lively in bed, / and graceto outlive those that we marry. / And also I pray Jesus to shorten the lives / of those that won't be governed by their wives”.


\section{Conclusion}
\label{Conclusion}

"The Wife of Bath's Tale" is one of the few Canterbury Tales entirely filled and concerned with women; the male character's job is merely to serve an ear piece for the women's desires and concerns. Although the land in which the tale is set is superficially ruled over by King Arthur, the king immediately yields control over a criminal knight's fate to his queen and her ladies (Bugge). These women drive the plot of the tale by determining and enforcing the knight's punishment. 

Also, by insisting that the Knight follows through with his promise to the loathly lady, by giving her anything she desires.

This is why Chaucer changed the original story “The Wedding of Sir Gawain and the Dame Ragnell” and “Tale of Florent” in “Confessio Amantisin” in his retelling through “The Wife of Bath’s Tale.” Chaucer needed the story to fit the character he had created. These alterations of the story fit far better with the tale The Wife of Bath is telling. This is why only after an in depth observation of the general plot, changes and the origins of the story can we begin to draw conclusions of the differences, and why they were made.




\bibliographystyle{ACM-Reference-Format}
\bibliography{reference}

\end{document}